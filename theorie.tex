\section{Theorie}
\label{sec:Theorie}

In diesem Experiment wird Röntgenstrahlung untersucht. Diese entsteht, wenn beschleunigte Elektronen
auf eine Anode auftreffen. In dieser werden die Elektronen durch das Coulombfeld des Atomkerns abgebremst und 
senden ein Photon aus, dessen Energie dem Energieverlust des Elektrons entspricht. Dabei entsteht ein 
kontinuierliches Bremsspektrum. Die minimale Wellenlänge bzw. maximale Energie entspricht 
\begin{equation*}
    \lambda_{\symup{min}} = \frac{\symup{ch}}{\symup{e_0}U} \; .
\end{equation*}
Bei dieser wird die ganze kinetische Energie in Strahlungsenergie $E = \symup{h}\nu$ umgewandelt.

Das charakteristische Spektrum entsteht jedoch durch die Ionisierung des Anodenmaterials. Bei der Ionisierung 
entsteht in einer inneren Schale eine Leerstelle, die mit einem Elektron der äußeren Schale wieder gefüllt wird.
Bei em Schalenübergang wird ein Photon der Energie $E = E_{\symup{m}} - E_{\symup{n}}$ frei. Die Linien werden 
mit $K_{\symup{\alpha}}$, $K_{\symup{\beta}}$, $L_{\symup{\alpha}}$ usw. benannt, wobei K, L, M usw. die 
Schale bezeichnen, auf der die Übergänge enden. Der griechische Buchstabe gibt an, von welcher Schale 
das Elektron kommt.

In einem Mehrelektronenatom schrimen die Hüllenelektronen und die Wechselwirkung der Elektronen untereinander 
die Kernladung ab, sodass sich die Coulombanziehung verringert und für die Bindungsenergie eines Elektrons 
auf der n-ten Schale gilt 
\begin{equation*}
    E_{\symup{n}} = - R_{\inf}z_{\symup{eff}}^2 \cdot \frac{1}{\symup{n}^2} \;  ,
\end{equation*}
wobei $z_{\symup{eff}} = z - \sigma$ die effektive Kernladung, $\sigma$ die Abschirmkonstante und $R_{\inf} = 13.6 \, \unit{\eV}$
die Rydbergenergie sind.

Die äußeren Elektronen besitzen durch ihren Bahndrehimpuls und des Elektronenspins unterschiedliche Bindungsenergien.
Das führt zu einer Feinstruktur von eng beieinander liegenden Linien.

\subsection{Absorption von Röntgenstrahlung}
Bei der Absorption von Röntgenstrahlung unter $1\,\text{MeV}$ dominieren der Comptoneffekt und der Photoeffekt.
Bei zunehmender Energie nimmt der Absorptionskoeffizient ab, steigt aber sprunghaft an, wenn die Photonenenergie 
gerade größer als die Bindungsenergie eines Elektrons der nächsten inneren Schale ist. 
Die Absorptionskanten sind genau durch oben 
genannte Sprünge charakterisiert. Ihre Energie $E = E_{\symup{n}} - E_{\inf}$ ist nahezu identisch mit der 
Bindungsenergie , welche der Sommerfeldschen Feinstrukturformel folgt
\begin{equation*}
    E_{n, j}=-R_{\infty}\left(z_{e f f, 1}^2 \cdot \frac{1}{n^2}+\alpha^2 z_{e f f, 2}^4 \cdot \frac{1}{n^3}\left(\frac{1}{j+\frac{1}{2}}-\frac{3}{4 n}\right)\right) \; .
\end{equation*}
Dabei ist $\alpha$ die Sommerfeldsche Feinstrukturkonstante, n die Hauptquantenzahl und j der Gesamtdrehimpuls.
Die Abschirmkonstante $\sigma_{\symup{K,abs}}$ der K-Kante berechnet sich durch 
\begin{equation*}
    \sigma_{\symup{K}} = Z - \sqrt{\frac{E_{\symup{K}}}{R_{\inf}}-\frac{\alpha^2Z^4}{4}}\; ,
\end{equation*}
wobei Z die Ordnungszahl ist.
Bei er L-Kante ist die Bestimmung der Abschirmkonstante jedoch komplizierter, das die Abschirmzahlen jedes 
beteiligten Elektrons berücksichtigt werden muss. Da die $L_{I}$- und $L_{II}$-Kante nicht aufgelöst werden 
können, vereinfacht sich die Relation mit Hilfe der Energiedifferenzen $\Delta E_{\symup{L}}= E_{\symup{L}_{II}} - E_{\symup{L}_{III}}$ zu 
\begin{equation*}
    \sigma_L=Z-\left(\frac{4}{\alpha} \sqrt{\frac{\Delta E_L}{R_{\infty}}}-\frac{5 \Delta E_L}{R_{\infty}}\right)^{1 / 2}\left(1+\frac{19}{32} \alpha^2 \frac{\Delta E_L}{R_{\infty}}\right)^{1 / 2}
\end{equation*}

\subsection{Bragg'sche Reflexion}
Experimentell kann die Wellenlänge der Röntgenstrahlung über die Bragg'sche Reflexion bestimmt werden.
Dabei wird das Beugungsmuster von Röntgenstrahlung, welche auf ein dreidimensionales Gitter fällt untersucht. 
Die Beugungsmaxima deuten dabei auf konstruktive Interferenz hin und folgen der Bragg'schen Bedingung
\begin{equation*}
    2d \sin \theta = \symup{n} \lambda \; .
\end{equation*}
Dabei ist $\theta$ der Glanzwinkel und n die Beugungsordnung.